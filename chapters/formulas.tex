\chapter{数学公式}

\section{Inline 公式}

由于输入便捷, \verb|$...$| 更受到欢迎.

\subsection{{\ttfamily math} 环境}

使用 {\ttfamily math} 环境:

\inputlatexcode{snippets/formulas/inline-quad1.tex}

\subsection{{\ttfamily \textbackslash(...\textbackslash)} 环境}

使用 \verb|\(...\)| 环境:

\inputlatexcode{snippets/formulas/inline-quad2.tex}

\subsection{{\ttfamily \$...\$} 环境}

使用 \verb|$...$| 环境:

\inputlatexcode{snippets/formulas/inline-quad3.tex}

\section{Displayed 公式}

推荐使用 \verb|\[...\]|, 在处理间距更好一些.

\subsection{{\ttfamily displaymath} 环境}

使用 {\ttfamily displaymath} 环境:

\inputlatexcode{snippets/formulas/displayed-quad1.tex}

\subsection{{\ttfamily \textbackslash[...\textbackslash]} 环境}

使用 \verb|\[...\]| 环境:

\inputlatexcode{snippets/formulas/displayed-quad2.tex}

\subsection{{\ttfamily \$\$...\$\$} 环境}

使用 \verb|$$...$$| 环境:

\inputlatexcode{snippets/formulas/displayed-quad3.tex}

\section{自动编号: {\ttfamily equation} 环境}

{\ttfamily equantion,align} 等环境默认是自动编号的,如果不需要编号则使用 {\ttfamily equation*,align*} 等环境。
对应多行公式内容,如果不需要编号则使用 \verb|\\notag|。

\inputlatexcode{snippets/formulas/numbering-equations.tex}

\section{多行公式}

\subsection{{\ttfamily multline} 环境}

\inputlatexcode{snippets/formulas/amsmath-multline.tex}

\subsection{{\ttfamily gather} 环境}

\inputlatexcode{snippets/formulas/amsmath-gather.tex}

\subsection{{\ttfamily align} 环境}

\inputlatexcode{snippets/formulas/amsmath-align.tex}

\subsection{{\ttfamily split} 环境}

\inputlatexcode{snippets/formulas/amsmath-split.tex}

\subsection{{\ttfamily flalign} 环境}

\inputlatexcode{snippets/formulas/amsmath-flalign.tex}

\subsection{{\ttfamily alignat} 环境}

\inputlatexcode{snippets/formulas/amsmath-alignat.tex}

\inputlatexcode{snippets/formulas/amsmath-alignat-text.tex}

注意: \verb|&&|实际上是\verb|& &|。 这里增加了一列水平间距,使得文字能够左对齐。

\subsection{{\ttfamily algined, gathered, alignedat} 环境}

Used for an aligned block within a math environment. This can be displayed math or in-line math.

\inputlatexcode{snippets/formulas/amsmath-aligned.tex}

\section{文本}

To insert some text into a formula, standard LaTeX provides the \verb|\mbox| command. 
amsmath offers further commands:

\begin{description}
  \item \verb|\text{words}|: inserts text within a math formula. The size is adjusted according to the current math style, that is, \verb|\text| produces smaller text within subscripts or superscripts.
  \item \verb|\intertext{text}|: suspends the formula, the text follows in a separate paragraph, then the multi-line formula is resumed, keeping the alignment. Use it for longer text.
\end{description}

\inputlatexcode{snippets/formulas/intertext.tex}

\section{Cases}

\inputlatexcode{snippets/formulas/cases-left.tex}

\verb|rcase|是自定义的环境:

\begin{minted}{latex}
\newenvironment{rcase}
  {\left.\begin{aligned}}
  {\end{aligned}\right\rbrace}
\end{minted}

\inputlatexcode{snippets/formulas/cases-right.tex}

\begin{remark*}
\verb|\,| Include a thin space in math mode.
\end{remark*}
