\chapter{Troubleshooting}

\section{url地址中的 \# 符号}

url 地址中 \# 符号需要 使用 \verb|\#|

例如,应该将

\mintinline{latex}|\url{http://example.com/#fragment}|

修改为

\mintinline{latex}|\url{http://example.com/\#fragment}|

\section{标题中的脚注或其它命令}

出现错误:
\mintinline{latex}|! Argument of \@sect has an extra }.|

参考: 

\url{https://texfaq.org/FAQ-extrabrace}

\url{https://texfaq.org/FAQ-ftnsect}

Use a robust command in place of the one you are using, or to force your command to be robust by prefixing it with \verbum*{protect}.

或者 

\mint{latex}{\section[<toc-title>]{<title>}}

在 \verbum{<toc-title>} 中不要含有命令。


