\chapter{源代码排版}

\section{{\ttfamily verbatim} 环境与{\ttfamily \textbackslash verb} 命令}

{\ttfamily verbatim} 环境适合比较简单场合下使用,无法实现代码的高亮显示,如:

\begin{latexcode}{colorbox}
\begin{verbatim}
#include <stdio.h>

int main(int argc, char **argc) {
  printf("Hello, world\n");
  return 0;
}
\end{verbatim}
\end{latexcode}

\verb|\verb| 命令可以使用行内的代码显示\footnote{行内长串代码显示存在换行的问题,可以使用 {\ttfamily \textbackslash ttfamily} 作为替代方案。}
,如:\verb!\verb|message = "hello,world"|!。

实际上使用 \verb|{\ttfamily message = "hello,world"}| 也可以达到类似的效果;
并且 \verb|\verb| 命令不能出现章节标题之中。

\section{listings 包}

\subsection{{\ttfamily lstlisting} 环境}

\begin{latexcode}{colorbox}
\begin{lstlisting}[language=C,caption={C/C++ 语言},numbers=left]
  /* Hello, world */
  int main{int argc, char **argv} {
    printf("Hello, world\n");
    return 0;
  }
\end{lstlisting}
\end{latexcode}

\subsection{{\ttfamily \textbackslash lstinline} 命令}

\verb|\lstinline| 命令可以使用行内的代码显示,如:\verb!\lstinline|message = "hello,world"|!

\subsection{{\ttfamily \textbackslash lstinputlisting} 命令}

显示代码文件,文件路径为相对于主文件的路径:

\begin{latexcode}{colorbox}
\lstinputlisting[language=C,caption={hello.c 文件},numbers=left]{snippets/codelistings/helloworld.c}
\end{latexcode}

\subsection{自定义语言}

listings 包没有提供 JavaScript 语言高亮显示, 但是可以自定义(详见模板)。

\begin{latexcode}{colorbox}
\begin{lstlisting}[language=js,caption={JavaScript 语言}]
// 向控制台输出 'hello, world'
const message = 'hello, world'
console.log(message)
\end{lstlisting}
\end{latexcode}

其它的扩展见:
\href{https://tex.stackexchange.com/questions/224093/adding-keywords-to-existing-language-for-listings-package}{Adding Keywords to Existing Language for Listings Package}

\section{minted 包}

minted 包需要借助 \href{https://pygments.org/}{pygments} 来渲染代码。
minted 整体风格比 listings 略胜一筹。

\subsection{支持的语言和风格}

首先,安装 pygments:

\mint{bash}{pip3 install pygments}

查看支持的语言:

\mint{bash}|pygmentize -L lexers|

查查支持的风格:

\mint{bash}|pygmentize -L styles|

此外,可以通过网站
\href{https://pygments.org/demo/}{Pygments website}或
\href{https://thepythonguru.com/tools/pygments-demo/}{Syntax Highlighter}
来查看各种代码渲染的风格\footnote{注意:部分风格与一些特殊字符有冲突,比较可靠的风格是 xcode。}。

\subsection{{\ttfamily minted} 环境}

\begin{latexcode}{colorbox}
\begin{minted}{c}
int main{int argc, char **argv} {
  printf("Hello, world\n");
  return 0;
}
\end{minted}
\end{latexcode}

\subsection{{\ttfamily \textbackslash mint} 命令}

对于单行的代码,可以使用 \verb|\mint| 命令
\footnote{注意:不是在行内显示代码,而是另起一段显示。},如:
\verb!\mint{js}|message = "hello, world"|!。

\subsection{{\ttfamily \textbackslash mintinline} 命令}

可以使用 \verb|\mintinline| 命令来在行内显示代码,如:
\verb!\mintinline{js}|message = "hello, world"|!。

\subsection{{\ttfamily \textbackslash inputminted} 命令}

显示代码文件,文件路径为相对于主文件的路径:

\begin{latexcode}{colorbox}
\inputminted{c}{snippets/codelistings/helloworld.c}
\end{latexcode}

\subsection{{\ttfamily listing} 环境}

minted 提供 {\ttfamily listing} 环境来作为一个浮动体来包裹代码。

\begin{latexcode}{colorbox}
\begin{listing}[H]
  \caption{hello.c文件}
  \inputminted{c}{snippets/codelistings/helloworld.c}
\end{listing}
\end{latexcode}

\subsection{Troubleshooting}

\subsubsection{在设置{\ttfamily bgcolor}后{\ttfamily \textbackslash mintinline}不换行}

在设置{\ttfamily bgcolor}后,对于较长的一行代码使用{\ttfamily \textbackslash mintinline}时不换行,
见
\href{https://github.com/gpoore/minted/issues/194}{Github issue 194: breaklines doesn't work with mintinline when other options are set}
和 
\href{https://tex.stackexchange.com/questions/419934/breaklines-doesnt-work-with-mintinline}{Stackoverflow: breaklines doesn't work with mintinline}。

The documentation mentions this as a limitation of bgcolor. The standard ways to put a background behind inline text don't work with line wrapping.

\subsubsection{代码的换页}

见
\href{https://tex.stackexchange.com/questions/368864/pagebreak-for-minted-in-figure}{Stackoverflow: Pagebreak for minted in figure}
和
\href{https://tex.stackexchange.com/questions/12428/code-spanning-over-two-pages-with-minted-inside-listing-with-caption/53540#53540}{Code spanning over two pages with minted, inside listing with caption}。

\subsubsection{一些字符出现红色方框}

见
\href{https://tex.stackexchange.com/questions/343494/minted-red-box-around-greek-characters}{Minted red box around greek characters}
和 
\href{https://tex.stackexchange.com/questions/424421/code-validation-in-minted-package-how-to-disable-it}{Code validation in minted package? How to disable it?}。

解决方法,使用下面的风格:
xcode,igor,rrt。

\section{{\ttfamily latexcode} 环境与{\ttfamily \textbackslash inputlatexcode} 命令}
为了显示{\LaTeX}代码和现实渲染结果
\footnote{实现类似{\ttfamily tcblisting}环境的功能},可以采用:
\begin{compactitems}
  \item {\ttfamily latexcode} 环境:工作过程是将代码输出到文件,读取并渲染该文件
  \item {\ttfamily \textbackslash inputlatexcode} 命令:直接读取并渲染该文件
\end{compactitems}

显示并渲染代码:

\begin{minted}{tex}
\begin{latexcode}{colorbox}
{\LaTeX} is a high-quality typesetting system; it includes features designed for the production of technical and scientific documentation.
\end{latexcode}
\end{minted}

显示并渲染代码文件,渲染结果带有 colorbox(这是默认参数):

\begin{minted}{tex}
\inputlatexcode[colorbox]{.tex文件路径}
\end{minted}

显示并渲染代码文件,渲染结果不带 colorbox
\footnote{algorithm 与 tcolorbox 存在冲突,请使用 nobox 参数}:

\begin{minted}{tex}
\inputlatexcode[nobox]{.tex文件路径}
\end{minted}



