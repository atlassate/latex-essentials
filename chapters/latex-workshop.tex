\chapter{{\LaTeX} Workshop 编辑器配置}

{\TeX} 自带编辑器 {\TeX} Works, 而使用 \href{https://code.visualstudio.com/}{Visulal Studio Code} +
\href{https://github.com/James-Yu/LaTeX-Workshop}{{\LaTeX} Workshop 插件}也是非常方便的.

\section{VSCode settings.json 文件}

To open the User settings:

\begin{enumerate}
    \item Open the command palette (either with F1 or Ctrl+Shift+P)
    \item Type "open settings"
\end{enumerate}

You are presented with two options, choose Open Settings (JSON)
Which, depending on platform, is one of:
\begin{itemize}
    \item Windows \url{\%APPDATA\%\\Code\\User\\settings.json}
    \item macOS \url{$HOME/Library/Application Support/Code/User/settings.json}
    \item Linux \url{$HOME/.config/Code/User/settings.json}
\end{itemize}

The Workspace settings will be in a {workspaceName}.code-workspace file where you saved it, and the Folder settings will be in a .vscode folder if and when it has been created.

\begin{minted}{js}
"latex-workshop.latex.recipe.default": "lastUsed",
"latex-workshop.latex.recipes": [
    {
        "name": "latexmk",
        "tools": [
            "latexmk"
        ]
    },
    {
        "name": "xelatex and bibtex",
        "tools": [
            "xelatex",
            "bibtex",
            "xelatex",
            "xelatex"
        ]
    }
],
"latex-workshop.latex.tools": [
    {
        "name": "latexmk",
        "command": "latexmk",
        "args": [
            "-xelatex",
            "-synctex=1",
            "-interaction=nonstopmode",
            "-file-line-error",
            "-shell-escape",
            "%DOC%"
        ]
    },
    {
        "name": "xelatex",
        "command": "xelatex",
        "args": [
            "-synctex=1",
            "-interaction=nonstopmode",
            "-file-line-error",
            "-shell-escape",
            "%DOC%"
        ],
        "env": {}
    },
    {
        "name": "bibtex",
        "command": "bibtex",
        "args": [
            "%DOCFILE%"
        ],
        "env": {}
    }
],
"latex-workshop.latex.autoClean.run": "onBuilt",
"latex-workshop.latex.clean.fileTypes": [ 
    "*.aux", "*.bbl", "*.pyg",
    "*.idx", "*.ind", "*.lof", 
    "*.lot", "*.out", "*.toc", 
    "*.acn", "*.acr", "*.alg", 
    "*.glg", "*.glo", "*.gls", 
    "*.fls", "*.log", "*.fdb_latexmk", 
    "*.snm", "*.synctex", "*.synctex.gz", 
    "*.nav", "*.xdv", "*.ilg",
    "*.blg", "*.lol", "*.cpt"
],
\end{minted}

\section{Build with Recipe}

Build with recipe:

macOS: fn+F1 -> Search 'Build with recipe' -> Select the recipe

默认的 recipe 是使用 latexmk, 为了处理中文, 配置编译器为 xelatex.
对于普通的文档可行, 但是遇到有文献目录的就无法自动生成.

含有文献目录的 tex 需要多次编译:

\begin{minted}{bash}
xelatex [options] filename.tex # 生成 filename.aux
bibtex filename.aux            # 生成 filename.bbl
xelatex [options] filename.tex
xelatex [options] filename.tex
\end{minted}

关于中间文件的删除

\begin{enumerate}
    \item 设置 onBuilt 就清除, 或者
    \item macOS: option+command+c 不太灵敏\footnote{快捷键查看: VSCode左边最下面齿轮 -> Keyboard Shortcuts}
\end{enumerate}
