\chapter{{\TeX} 的安装与配置}

\section{{\TeX} 版本}

{\TeX} 官方版本是 \href{http://www.tug.org/texlive}{{\TeX} Live}, 还有各种发行版
\footnote{中文的 CTEX 套装已不再更新}: Windows 平台的 MiKTeX 和 macOS 平台的 MacTex. 
后面几个版本不如官方版本稳定, 推荐安装官方版本.

\begin{itemize}
    \item{在 Debian 中安装 {\TeX} Live}

    不要使用 apt 安装, 这个安装的是 Debian 社区维护的版本, 使用 tlmgr 时会出现一些问题, 
    不能自动连到最新的仓库会报错, 如: Debian 10 对应的是 texlive 2018, 
    而 \href{https://tug.org/}{TUG} 最新的仓库对应的是 texlive 2019.

    \item{在 macOS 中安装 {\TeX} Live}

    尽管 \href{https://tug.org/}{TUG} 推荐安装 MacTex, 但是使用 install-tl 安装脚本可以保证与 Linux 安装一致. 
\end{itemize}

\section{安装 {\TeX}}

\subsection{下载安装脚本}

官方主页: \url{https://www.ctan.org/}, 国内镜像网站\footnote{北京交通大学镜像比清华大学镜像快.}:

\begin{itemize}
    \item \url{https://mirror.bjtu.edu.cn/CTAN/systems/texlive/tlnet}
    \item \url{https://mirrors.aliyun.com/CTAN/systems/texlive/tlnet}
    \item \url{http://mirrors.cloud.tencent.com/CTAN/systems/texlive/tlnet}
\end{itemize}

国内从 \href{https://mirror.bjtu.edu.cn/CTAN/systems/texlive/tlnet}{北京交通大学镜像} 下载较快, 
下载其中的 install-tl.zip, 并解压缩, 该文件包含了各平台(Linux, macOS, Windows)的安装脚本.

注意: 不能只下载 install-tl 运行, 否则出错:
\begin{minted}{text}
Can't locate TeXLive/TLUtils.pm in @INC (you may need to install the TeXLive::TLUtils module) (@INC contains: ./tlpkg /usr/local/Cellar/perl/5.32.1_1/lib/perl5/site_perl/5.32.1/darwin-thread-multi-2level /usr/local/Cellar/perl/5.32.1_1/lib/perl5/site_perl/5.32.1 /usr/local/Cellar/perl/5.32.1_1/lib/perl5/5.32.1/darwin-thread-multi-2level /usr/local/Cellar/perl/5.32.1_1/lib/perl5/5.32.1 /usr/local/lib/perl5/site_perl/5.32.1) at ./install-tl line 150.
BEGIN failed--compilation aborted at ./install-tl line 154.
\end{minted}

使脚本成为可执行文件:

\begin{minted}{bash}
chmod +x install-tl
\end{minted}

\subsection{创建目录}

\begin{minted}{bash}
sudo mkidr /usr/local/texlive
sudo chown USERNAME /usr/local/texlive
\end{minted}

\subsection{通过图形界面安装}

运行 install-tl 脚本, 为了避免 {\LaTeX} 在编译 tex 文件时报错而临时下载各种包, 
建议安装过程中选择完整安装 full scheme, 占用磁盘空间约 8G.

\begin{minted}{bash}
./install-tl --repository https://mirror.bjtu.edu.cn/CTAN/systems/texlive/tlnet
\end{minted}

\subsection{通过命令行安装 }

同样也可以通过命令行安装:

\begin{minted}{bash}
$ ./install-tl --no-gui --repository https://mirror.bjtu.edu.cn/CTAN/systems/texlive/tlnet
\end{minted}

\subsection{添加路径到 PATH}

安装结束后将 {\TeX}  Live 命令行目录加到 PATH, macOS 修改 \url{~/.bash_profile};
Debian 修改 \url{/etc/profile} (对所有用户生效):

\begin{itemize}

    \item  2020 版 for macOS
    \mint{bash}|export PATH=/usr/local/texlive/2020/bin/x86_64-darwin:$PATH|

    \item 2021 版 for macOS(Apple 发布了 M1 处理器)
    \mint{bash}|export PATH=/usr/local/texlive/2021/bin/universal-darwin:$PATH|

    \item Debian
    \mint{bash}|export PATH=/usr/local/texlive/2021/bin/x86_64-linux:$PATH|
\end{itemize}

\section{CTAN 镜像使用帮助}

CTAN (The Comprehensive TeX Archive Network) 镜像源可以使用 {\TeX} Live 管理器 tlmgr 更改.

在命令行中执行:
\mint{bash}|tlmgr option repository https://mirrors.tuna.tsinghua.edu.cn/CTAN/systems/texlive/tlnet|
即可永久更改镜像源.

如果只需要临时切换, 可以用如下命令:
\mint{bash}|tlmgr update --all --repository https://mirrors.tuna.tsinghua.edu.cn/CTAN/systems/texlive/tlnet|
其中的 \mintinline{bash}|update --all| 指令可根据需要修改.
