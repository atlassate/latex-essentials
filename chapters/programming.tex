\chapter{{\LaTeX 编程}}

\section{新建命令 {\ttfamily newcommand}}

\section{{\ttfamily def}}

\section{新建环境 {\ttfamily newenvironment}}

\section{key-value 参数}

关于 key-value 参数的实现 请参考 Joseph Wright 和 Christian Feuersänger 的文章\cite{TUGBOAT2009}。

\subsection{{\ttfamily \textbackslash define@key}}

\mint{latex}|\define@key{<family>}{<key>}{<handler>}|

{\ttfamily \textbackslash define@key} 有三个参数:{\ttfamily <family>},
{\ttfamily <key>} 和 {\ttfamily <handler>}。

\subsection{{\ttfamily \textbackslash setkeys}}

{\ttfamily \textbackslash setkeys} 用于设置 key-value。The input to {\ttfamily \textbackslash setkeys} is a comma-separated list: each comma-separated {\ttfamily <key>=<value>} pair is therefore processed in turn. 

\subsection{ keyval 包}

下面的例子由 keyval 文档示例修改得到:

\inputminted[linenos]{latex}{examples/programming/keyval-commands-with-parameters.tex}

编译时在终端输出:

\begin{minted}{text}
Setting height (4 in)
Setting width (5 in)
Setting scale (.85)
Setting bounding box (20 20 300 400)
Setting clip (false)
Lorem ipsum dolor sit amet
Setting height (4 in)
Setting width (5 in)
Setting clip (true)
Lorem ipsum dolor sit amet
\end{minted}

从上面的输出可以看出,当参数列表未包含的参数时不会调用相关的{\ttfamily <handler>};
对于有默认值的参数,可以不给值,但是需要列出该参数。

为了使键值对参数时可选的,我们可以将值进行存储,如下例:

\inputminted[linenos]{latex}{examples/programming/keyval-commands-with-optional-parameters1.tex}

其中定义命令的方式也可以采用 {\ttfamily newcommand}:

\inputminted[firstline=30,lastline=42]{latex}{examples/programming/keyval-commands-with-optional-parameters2.tex}



参考\cite{WIKIBOOKS}

