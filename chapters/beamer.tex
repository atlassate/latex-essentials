\chapter{Beamer 演示文稿}

\section{Beamer 主题与颜色}

\begin{itemize}
  \item Beamer 内置主题与颜色预览 \href{https://mpetroff.net/files/beamer-theme-matrix/}{Beamer Theme Matrix}
  \item Metropolis 主题 \href{https://github.com/matze/mtheme}{Beamer Theme Metropolis}
  \item Solarized 颜色 \href{https://github.com/jrnold/beamercolorthemesolarized}{Beamer Color Theme Solarized}
\end{itemize}

Beamer 中的 \mintinline{latex}|\pause, \pausesections| 和列表动画会产生额外的页面, 这点在打印文稿时需注意.

\section{设计演示文稿}

设计演示文件时应注意:

\begin{itemize}
  \item Keep time constraints in mind; a frame per minute is a good rule of thumb.
  \item Use few sections, logically split in subsections; it is better to avoid subsubsections.
  \item Use self-explanatory titles for sectioning and frames.
  \item Bulleted lists help to keep things simple.
  \item Consider avoiding numbering references; one rarely cares about a
reference to theorem 2.6 during a talk.
  \item Don't disrupt the reading flow with footnotes.
  \item Graphics, such as diagrams, help the audience with visualization.   
  \item Slides should support your talk, not the other way round. Did you
already bear with a presentation where the speaker just read aloud the text from the slides and used fancy transition effects? You can do it better.
\end{itemize}
