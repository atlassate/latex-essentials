\chapter{定理与证明的排版}



\section{使用 {\AmS} amsthm 包}

在 compositors 宏包中定义了 {\ttfamily theorem,
lemma,
proposition,
corollary,
definition,
example,
remark,
remark*,
solution,
solution*,
proof}环境。

\begin{latexcode}{colorbox}
\begin{definition}[Hypotenuse]
  The longest side of a triangle with a right angle is called the \emph{hypotenuse}.
\end{definition}
\end{latexcode}

\begin{latexcode}{colorbox}
\begin{remark}
  The other sides are called \emph{catheti}, or \emph{legs}.
\end{remark}
\end{latexcode}

\begin{latexcode}{colorbox}
\begin{theorem}[Pythagoras]
  \label{pythagoras}
  In any right triangle, the square of the hypotenuse equals the sum of the squares of the other sides.
\end{theorem}
\end{latexcode}

\begin{latexcode}{colorbox}
\begin{proof}
  The proof has been given in Euclid's Elements,
  Book 1, Proposition 47. Refer to it for details.
  The converse is also true, see lemma \ref{converse}. 
\end{proof}
\end{latexcode}

\begin{latexcode}{colorbox}
\begin{lemma}
  \label{converse}
  For any three positive numbers \(x\), \(y\),
  and \(z\) with \(x^2 + y^2 = z^2\), there is a triangle with side lengths \(x\), \(y\) and \(z\). Such triangle has a right angle, and the hypotenuse has the length \(z\).
\end{lemma}
\end{latexcode}

\begin{latexcode}{colorbox}
\begin{remark*}
  This is the converse of theorem \ref{pythagoras}.
\end{remark*}
\end{latexcode}

\section{例题与解的排版}

\begin{latexcode}{colorbox}
\begin{example}
  求方程 $$ x^2 - x - 12 = 0 $$ 的根.
\end{example}
  
\begin{solution*}
  对方程左边进行因式分解:
  \begin{gather*}
    (x-4)(x+3) = 0 \\
    \intertext{因此,}
    x-4=0 \text{\quad 或 \quad} x+3=0 \\
    \intertext{所以, 方程的根是:}
    x_1 = 4, \quad x_2 = -3
  \end{gather*}
\end{solution*}
\end{latexcode}