\chapter{算法排版}

{\LaTeX}中算法或伪代码的排版可以选择下面两种方案的一种
\footnote{\url{https://www.overleaf.com/learn/latex/Algorithms}}:

\begin{compactitems}
  \item 使用 algpseudocode 或 algcompatible 或 algorithmic包(三选一)排版算法主体部分,使用 algorithm 包设置算法标题
  \item 使用 algorithm2e 包
\end{compactitems}

注意:上述宏包不能同时使用,否则将产生很多错误和冲突。各宏包的主要功能
\footnote{\url{https://tex.stackexchange.com/questions/229355/algorithm-algorithmic-algorithmicx-algorithm2e-algpseudocode-confused}}:
\begin{description}
  \item [algorithm] float wrapper for algorithms.
  \item [algorithmic] first algorithm typesetting environment.
  \item [algorithmicx] second algorithm typesetting environment.
  \item [algpseudocode] layout for algorithmicx.
  \item [algorithm2e] third algorithm typesetting environment.
\end{description}

\section{伪代码与 Pascal 代码风格的算法}

在导言区加入:
\begin{minted}{latex}
\usepackage{algorithm}
\usepackage{algpseudocode}
\usepackage{algpascal}
\end{minted}

\inputlatexcode[nobox]{snippets/algorithms/euclidean-algorithm1.tex}

\inputlatexcode[nobox]{snippets/algorithms/euclidean-algorithm2.tex}

\begin{remark*}
使用时注意以下问题:
\begin{enumerate}
  \item 算法排版, algorithm 与 tcolorbox 存在冲突, 即 {\ttfamily tcolorbox} 环境不能包含 {\ttfamily algorithm} 环境;
  \item 当同时引入 algpseudocode 和 algpascal, 必须使用 \verb|\aglanguage{pseudocode}| 或 
  \verb|\aglanguage{pascal}|, 否则不能正确解析数学公式.
\end{enumerate}
\end{remark*}

\section{跨页显式算法}

如果算法较长, 则需要分页显示, 有两种方法:
\begin{enumerate}
  \item \verb|\agstore| 和 \verb|\agrestore| 命令将算法分成几个部分
  \item 自定义 {\ttfamily breakablealgorithm} 环境
  \footnote{\url{https://tex.stackexchange.com/questions/33866/algorithm-tag-and-page-break}}
\end{enumerate}

\subsection{{\ttfamily \textbackslash algstore} 和 {\ttfamily \textbackslash algrestore} 命令}

\inputlatexcode[nobox]{snippets/algorithms/breakable-algorithm1.tex}

\subsection{{\ttfamily breakablealgorithm} 环境}

\inputlatexcode[nobox]{snippets/algorithms/breakable-algorithm2.tex}