The quadratic equation
\begin{equation}
  ax^2 + bx + c = 0,
  \label{quad}
\end{equation}
where \(a, b\) and \( c \) are constants and \( a \neq 0 \),
has two solutions for the variable \( x \):
\begin{equation}
  \label{root}
  x_{1,2} = \cfrac{-b \pm \sqrt{b^2 - 4ac}}{2a}.
\end{equation}

If the \emph{discrimimant} \( \Delta \) with
\[
\Delta = b^2 - 4ac 
\]
is zero, then the equation (\ref{quad}) has a double solution: (\ref{root}) becomes
\[
x = - \frac{b}{2a}. 
\]