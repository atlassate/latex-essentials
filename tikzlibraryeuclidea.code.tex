\ProvidesFile{tikzlibraryeuclidea.code.tex}[2023/01/01 v1.0.0 A tikz library for plane geometry]

% 注意: 计算过程是保留坐标单位(pt)的, 所以存在乘除法单位的问题, 首先数值始终携带单位, 在 calc 运算时有的需要
% 转换为标量; 将坐标转换为 pt 值, 数值可能超出限值, 出现 Dimension too large 错误, 
% 在计算长度时及时进行缩小 https://tex.stackexchange.com/questions/475556/tikz-why-is-dimension-too-large
% 具体方法是修改默认的 1cm, 如: [scale=1.0,x=0.5cm,y=0.5cm]
% 注意此处的变量不要和 tikzpicture 环境重名, 否则被替换掉
% triangle centers: https://mathworld.wolfram.com/BarycentricCoordinates.html
\tikzmath{
  % 采用列主元消元法求直线 P1P2 与直线 P3P4 的交点 P 位置参数 s: s = P1P/P1P2
  function intersectll(\x1,\y1,\x2,\y2,\x3,\y3,\x4,\y4) {
    \a1 = \x2-\x1; \b1 = \x3-\x4; \c1 = \x3-\x1;
    \a2 = \y2-\y1; \b2 = \y3-\y4; \c2 = \y3-\y1;
    \dmax = max(max(abs(\a1),abs(\a2)), max(abs(\b1),abs(\b2)));
    \a1 = \a1/\dmax; \b1 = \b1/\dmax; \c1=\c1/\dmax;
    \a2 = \a2/\dmax; \b2 = \b2/\dmax; \c2=\c2/\dmax;
    if abs(\a1) < abs(\a2) then {
      \temp = \a1; \a1 = \a2; \a2 = \temp;
      \temp = \b1; \b1 = \b2; \b2 = \temp;
      \temp = \c1; \c1 = \c2; \c2 = \temp;
    };
    \b1 = \b1/\a1; \c1 = \c1/\a1; \a1 = 1.0;
    \b2 = \b2-\a2*\b1; \c2 = \c2-\a2*\c1; \a2 = 0.0;
    \n2 = \c2/\b2; \n1 = \c1-\b1*\n2;
    return \n1;
  };
}

\tikzset{
  % perpendicular bisector of the line segment (#1 -- #2)
  % specifying start and end with modifiers(#3 and #4,see tikz manual 13.5) 
  perpendicular bisector/.style args={#1,#2,#3,#4}{
    insert path={
      let
        \p1=($(#1)!0.5!(#2)$),
        \p2=($(\p1)!#3!90:(#2)$), 
        \p3=($(\p1)!#4!-90:(#2)$)  
      in (\p2) -- (\p3)
    }
  },
  % perpendicular line of the line (#1 -- #2) through #3
  % specifying start and end with modifiers(#4 and #5,see tikz manual 13.5) 
  perpendicular/.style args={#1,#2,#3,#4,#5}{
    insert path={
      let
        \p1=($(#1)!(#3)!(#2)$),   
        \p2=($(#3)!#4!(\p1)$),    
        \p3=($(#3)!#5!(\p1)$)     
      in (\p2) -- (\p3)
    }
  },
  % parallel line of the line (#1 -- #2) through #3
  % specifying start and end with modifiers(#4 and #5,see tikz manual 13.5) 
  parallel/.style args={#1,#2,#3,#4,#5}{
    insert path={
      let
        \p1=($(#3)+(#2)-(#1)$),    
        \p2=($(#3)!#4!(\p1)$),    
        \p3=($(#3)!#5!(\p1)$)    
      in (\p2) -- (\p3)
    }
  },
  % translate #3 by the vector (#1 -> #2)
  translate/.style args={#1,#2,#3}{
    insert path={
      let
        \p1=($(#3)+(#2)-(#1)$),    
      in (\p1)
    }
  },
  % reflect #3 about the line (#1 -- #2) 
  reflect/.style args={#1,#2,#3}{
    insert path={
      let
        \p1=($(#1)!(#3)!(#2)$),
        \p2=($(#3)!2!(\p1)$),    
      in (\p2)
    }
  },
  % rotate #2 around #1 with degree #3
  rotate/.style args={#1,#2,#3}{
    insert path={
      let 
        \p1=(#1),
        \p2=(#2),
      in ($(\p1)!1!#3:(\p2)$)
    }
  },
  % intersection of angle bisector and the opposite side to the vertex (#1)
  % (using the angle bisector theorem)
  angle bisector/.style args={#1,#2,#3}{
    insert path={
      let 
        \p1 = (#1), \p2 = (#2), \p3 = (#3),
        \p{AB} = ($(#2)-(#1)$),
        \p{AC} = ($(#3)-(#1)$),
        \n{c} = {veclen(\x{AB}, \y{AB})},
        \n{b} = {veclen(\x{AC}, \y{AC})},
        \n{r} = {scalar(\n{c}/(\n{b}+\n{c}))}, % 转换为标量
      in ($(\p2)!\n{r}!(\p3)$)
    }
  },
  angle/.style args={#1,#2,#3,#4,#5,#6}{
    insert path={
      let 
        \p1 = (#1), \p2 = (#2), \p3 = (#3),
        \p4 = (#4), \p5 = (#5), \n1 = {#6},
        \p{AB} = ($(#2)-(#1)$),
        \p{BC} = ($(#3)-(#2)$),
        \p{CA} = ($(#1)-(#3)$),
        \p{PQ} = ($(#5)-(#4)$),
        \n{a} = {veclen(\x{BC}, \y{BC})},
        \n{b} = {veclen(\x{CA}, \y{CA})},
        \n{c} = {veclen(\x{AB}, \y{AB})},
        \n{s} = {\n{a}+\n{b}+\n{c}},
        \n{a} = {\n{a}/\n{s}},  % Avoid 'Dimension too large'
        \n{b} = {\n{b}/\n{s}},
        \n{c} = {\n{c}/\n{s}},
        \n{theta} = {acos((\n{b}*\n{b}+\n{c}*\n{c}-\n{a}*\n{a})/(2*\n{b}*\n{c}))},
        \n{Cos} = {cos(\n1*\n{theta})},
        \n{Sin} = {sin(\n1*\n{theta})},
      in ({\x4+\x{PQ}*\n{Cos}-\y{PQ}*\n{Sin},\y4+\x{PQ}*\n{Sin}+\y{PQ}*\n{Cos}})
    }
  },
  % incenter coordinate 
  incenter/.style args={#1,#2,#3}{
    insert path={
      let
        \p1 = (#1), \p2 = (#2), \p3 = (#3),
        \p{AB} = ($(#2)-(#1)$),
        \p{BC} = ($(#3)-(#2)$),
        \p{CA} = ($(#1)-(#3)$),
        \n{a} = {veclen(\x{BC}, \y{BC})},
        \n{b} = {veclen(\x{CA}, \y{CA})},
        \n{c} = {veclen(\x{AB}, \y{AB})},
        \n{s} = {\n{a}+\n{b}+\n{c}},
        \n1 = {\n{a}/\n{s}}, 
        \n2 = {\n{b}/\n{s}},
        \n3 = {\n{c}/\n{s}},
      in ({\n1*\x1+\n2*\x2+\n3*\x3,\n1*\y1+\n2*\y2+\n3*\y3})
    }
  },
  % excenter coordinate corresponding to the vertex #1
  excenter/.style args={#1,#2,#3}{ 
    insert path={
      let
        \p1 = (#1), \p2 = (#2), \p3 = (#3),
        \p{AB} = ($(#2)-(#1)$),
        \p{BC} = ($(#3)-(#2)$),
        \p{CA} = ($(#1)-(#3)$),
        \n{a} = {veclen(\x{BC}, \y{BC})},
        \n{b} = {veclen(\x{CA}, \y{CA})},
        \n{c} = {veclen(\x{AB}, \y{AB})},
        \n{s} = {-\n{a}+\n{b}+\n{c}},
        \n1 = {\n{a}/\n{s}}, 
        \n2 = {\n{b}/\n{s}},
        \n3 = {\n{c}/\n{s}},
      in ({-\n1*\x1+\n2*\x2+\n3*\x3,-\n1*\y1+\n2*\y2+\n3*\y3})
    }
  },
  % circumcenter coordinate
  circumcenter/.style args={#1,#2,#3}{
    insert path={
      let
        \p1 = (#1), \p2 = (#2), \p3 = (#3),
        \p{AB} = ($(#2)-(#1)$),
        \p{BC} = ($(#3)-(#2)$),
        \p{CA} = ($(#1)-(#3)$),
        \n{a} = {veclen(\x{BC}, \y{BC})},
        \n{b} = {veclen(\x{CA}, \y{CA})},
        \n{c} = {veclen(\x{AB}, \y{AB})},
        \n{m} = {max(max(\n{a},\n{b}),\n{c})},
        \n{a} = {\n{a}/\n{m}},
        \n{a} = {\n{a}*\n{a}},
        \n{b} = {\n{b}/\n{m}},
        \n{b} = {\n{b}*\n{b}},
        \n{c} = {\n{c}/\n{m}},
        \n{c} = {\n{c}*\n{c}},
        \n1 = {\n{a}*(\n{b}+\n{c}-\n{a})},
        \n2 = {\n{b}*(\n{c}+\n{a}-\n{b})},
        \n3 = {\n{c}*(\n{a}+\n{b}-\n{c})},
        \n{s} = {\n1+\n2+\n3},
        \n1 = {\n1/\n{s}},
        \n2 = {\n2/\n{s}},
        \n3 = {\n3/\n{s}},
      in ({\n1*\x1+\n2*\x2+\n3*\x3,\n1*\y1+\n2*\y2+\n3*\y3})
    }
  },
  % orthocenter coordinate
  orthocenter/.style args={#1,#2,#3}{
    insert path={
      let
        \p1 = (#1), \p2 = (#2), \p3 = (#3),
        \p{AB} = ($(#2)-(#1)$),
        \p{BC} = ($(#3)-(#2)$),
        \p{CA} = ($(#1)-(#3)$),
        \n{a} = {veclen(\x{BC}, \y{BC})},
        \n{b} = {veclen(\x{CA}, \y{CA})},
        \n{c} = {veclen(\x{AB}, \y{AB})},
        \n{m} = {max(max(\n{a},\n{b}),\n{c})},
        \n{a} = {\n{a}/\n{m}},
        \n{a} = {\n{a}*\n{a}},
        \n{b} = {\n{b}/\n{m}},
        \n{b} = {\n{b}*\n{b}},
        \n{c} = {\n{c}/\n{m}},
        \n{c} = {\n{c}*\n{c}},
        \n{a2} = {\n{b}+\n{c}-\n{a}},
        \n{b2} = {\n{c}+\n{a}-\n{b}},
        \n{c2} = {\n{a}+\n{b}-\n{c}},
        \n1 = {\n{c2}*\n{b2}},
        \n2 = {\n{a2}*\n{c2}},
        \n3 = {\n{b2}*\n{a2}},
        \n{s} = {\n1+\n2+\n3},
        \n1 = {\n1/\n{s}},
        \n2 = {\n2/\n{s}},
        \n3 = {\n3/\n{s}},
      in ({\n1*\x1+\n2*\x2+\n3*\x3,\n1*\y1+\n2*\y2+\n3*\y3})
    }
  },
  % centroid coordinate
  centroid/.style args={#1,#2,#3}{
    insert path={
      let
        \p1 = (#1), \p2 = (#2), \p3 = (#3),
      in ({(\x1+\x2+\x3)/3},{(\y1+\y2+\y3)/3})
    }
  },
  % nine-pint center coordinate
  nine-point center/.style args={#1,#2,#3}{
    insert path={
      let
        \p1 = (#1), \p2 = (#2), \p3 = (#3),
        \p{AB} = ($(#2)-(#1)$),
        \p{BC} = ($(#3)-(#2)$),
        \p{CA} = ($(#1)-(#3)$),
        \n{a} = {veclen(\x{BC}, \y{BC})},
        \n{b} = {veclen(\x{CA}, \y{CA})},
        \n{c} = {veclen(\x{AB}, \y{AB})},
        \n{m} = {max(max(\n{a},\n{b}),\n{c})},
        \n{a} = {\n{a}/\n{m}},
        \n{a} = {\n{a}*\n{a}},
        \n{b} = {\n{b}/\n{m}},
        \n{b} = {\n{b}*\n{b}},
        \n{c} = {\n{c}/\n{m}},
        \n{c} = {\n{c}*\n{c}},
        \n1 = {\n{a}*(\n{b}+\n{c})-(\n{b}-\n{c})*(\n{b}-\n{c})},
        \n2 = {\n{b}*(\n{c}+\n{a})-(\n{c}-\n{a})*(\n{c}-\n{a})},
        \n3 = {\n{c}*(\n{a}+\n{b})-(\n{a}-\n{b})*(\n{a}-\n{b})},
        \n{s} = {\n1+\n2+\n3},
        \n1 = {\n1/\n{s}},
        \n2 = {\n2/\n{s}},
        \n3 = {\n3/\n{s}},
      in ({\n1*\x1+\n2*\x2+\n3*\x3,\n1*\y1+\n2*\y2+\n3*\y3})
    }
  },
  % intersection coordinate of the line (#1 -- #2) and the line (#3 -- #4) 
  % https://en.wikipedia.org/wiki/Line%E2%80%93line_intersection
  intersect/.style args={#1,#2,#3,#4}{
    insert path={
      let 
        \p1 = (#1), \p2 = (#2), \p3 = (#3), \p4 = (#4), 
        \n1 = {intersectll(\x1,\y1,\x2,\y2,\x3,\y3,\x4,\y4)},
      in ($(\p1)!{\n1}!(\p2)$)
    }
  },
  % circle with the center (#1) through #2
  circle/.style args={#1,#2}{
    insert path={
      let
        \p1 = (#1),
        \p2 = ($(#1)-(#2)$),
        \n1 = {veclen(\p2)}
      in (\p1) circle ({\n1})
    }
  },
  % 从圆[圆心 O(#1),圆上一点 A(#2)]外一点 P(#3) 作圆的切线, 
  % 返回切点为 T 的坐标, 点 T 在 OP 的逆时针方向
  tangent point/.style args={#1,#2,#3}{
    insert path={
      let 
        \p1 = (#1),
        \p2 = (#3),
        \p3 = ($(#2)-(#1)$), % 半径
        \p4 = ($(#3)-(#1)$), 
        \n1 = {veclen(\p3)},
        \n2 = {veclen(\p4)},
        \n3 = {scalar(\n1/\n2)}
      in ($(\p1)!\n3!{acos(\n1/\n2)}:(\p2)$)
    }
  },
  % 两圆的位似中心 Homothetic center
  % external center, #1,#2: 圆心1和圆1上任一点, #3,#4: 圆心2和圆2上任一点
  external center/.style args={#1,#2,#3,#4}{
    insert path={
      let 
        \p1 = (#1),
        \p2 = (#3),
        \p3 = ($(#2)-(#1)$), % 半径1
        \p4 = ($(#4)-(#3)$), % 半径2
        \n1 = {veclen(\p3)},
        \n2 = {veclen(\p4)},
        \n3 = {scalar(\n1/(\n1-\n2))}
      in ($(\p1)!\n3!(\p2)$)
    }
  },
  % 两圆的位似中心 Homothetic center
  % internal center, #1,#2: 圆心1和圆1上任一点, #3,#4: 圆心2和圆2上任一点
  internal center/.style args={#1,#2,#3,#4}{
    insert path={
      let 
        \p1 = (#1),
        \p2 = (#3),
        \p3 = ($(#2)-(#1)$), % 半径1
        \p4 = ($(#4)-(#3)$), % 半径2
        \n1 = {veclen(\p3)},
        \n2 = {veclen(\p4)},
        \n3 = {scalar(\n1/(\n1+\n2))}
      in ($(\p1)!\n3!(\p2)$)
    }
  },
}
% -----------------------------------------------------------------------------

