% Using pause command
\documentclass{beamer} 

\usetheme{Rochester} 

\begin{document} 
 
\begin{frame}{Pythagorean Theorem} 
 In mathematics, the \textbf{Pythagorean theorem} or \textbf{Pythagoras' theorem} is a fundamental relation in Euclidean geometry between the three sides of a right triangle. It states that the area of the square whose side is the hypotenuse (the side opposite the right angle) is equal to the sum of the areas of the squares on the other two sides. 
 
\pause 
 
This theorem can be written as an equation relating the lengths of the sides $a$, $b$ and the hypotenuse $c$, often called the \textbf{Pythagorean equation}:

\pause
\[ 
  a^2 + b^2 = c^2 
\] 
 
\pause 
 
A \textbf{Pythagorean triple} has three positive integers $a$, $b$, and $c$, such that $a^2 + b^2 = c^2$.
The following is a list of primitive Pythagorean triples with values less than 100:
(3, 4, 5), (5, 12, 13), (7, 24, 25), (8, 15, 17), (9, 40, 41), (11, 60, 61), (12, 35, 37), (13, 84, 85), (16, 63, 65), (20, 21, 29), (28, 45, 53), (33, 56, 65), (36, 77, 85), (39, 80, 89), (48, 55, 73), (65, 72, 97)
 
\end{frame} 
 
\end{document} 