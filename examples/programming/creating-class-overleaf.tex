% https://www.overleaf.com/learn/latex/Writing_your_own_class
% 创建文档类文件
\begin{filecontents}[overwrite]{exampleclass.cls}
\NeedsTeXFormat{LaTeX2e}
\ProvidesClass{exampleclass}[2014/08/16 Example LaTeX class]

\newcommand{\headlinecolor}{\normalcolor}
\RequirePackage{xcolor}
\definecolor{slcolor}{HTML}{882B21}

\DeclareOption{onecolumn}{\OptionNotUsed}
\DeclareOption{green}{\renewcommand{\headlinecolor}{\color{green}}}
\DeclareOption{red}{\renewcommand{\headlinecolor}{\color{slcolor}}}
\DeclareOption*{\PassOptionsToClass{\CurrentOption}{article}}
\ProcessOptions\relax
\LoadClass[twocolumn]{article}

\renewcommand{\maketitle}{%
  \twocolumn[%
    \fontsize{50}{60}\fontfamily{phv}\fontseries{b}%
    \fontshape{sl}\selectfont\headlinecolor
    \@title
    \medskip
    ]%
}

\renewcommand{\section}{%
  \@startsection
  {section}{1}{0pt}{-1.5ex plus -1ex minus -.2ex}%
  {1ex plus .2ex}{\large\sffamily\slshape\headlinecolor}%
}

\renewcommand{\normalsize}{\fontsize{9}{10}\selectfont}
\setlength{\textwidth}{17.5cm}
\setlength{\textheight}{22cm}
\setcounter{secnumdepth}{0}
\end{filecontents}

% 使用文档类
\documentclass[red]{exampleclass}
\usepackage[utf8]{inputenc}
\usepackage[english]{babel}

\usepackage{blindtext}

\title{Example to show how classes work}
\author{Overleaf}
\date{First created in August 2014}

\begin{document}

\maketitle

\noindent
Let's begin with a simple working example here.

\blindtext

\section{Introduction}

The Monty Hall problem is an interesting puzzle in mathematics inspired by the TV 
game show "Let's make a deal". It's a famous problem that is really easy to understand.

The problem became famous after its appearance in the column "Ask Marilyn" in 1990 and 
it's described below:

Suppose you're on a game show, and you're given the choice of three doors: Behind one 
door is a car; behind the others, goats. You pick a door, say No. 1, and the host, who 
knows what's behind the doors, opens another door, say No. 3, which has a goat. He then 
says to you, "Do you want to pick door No. 2?" Is it to your advantage to switch your
choice?


Let's try to figure it out. At this point there are two doors, behind one door is a goat 
and behind the other one is a car, thus the probability of choosing the right one is 50\%,
changing the decision doesn't make any difference in the chances of winning. It's pretty
much the same scenario of tossing a coin.

Even though the ideas exposed in the previous paragraph may seem correct, actually 
switching the choice increases the probability of winning. To understand why let's 
first check the facts:

\section{The same thing}

The Monty Hall problem is an interesting puzzle in mathematics inspired by the TV 
game show "Let's make a deal". It's a famous problem that is really easy to understand.

The problem became famous after its appearance in the column "Ask Marilyn" in 1990 and 
it's described below:

Suppose you're on a game show, and you're given the choice of three doors: Behind one 
door is a car; behind the others, goats. You pick a door, say No. 1, and the host, who 
knows what's behind the doors, opens another door, say No. 3, which has a goat. He then 
says to you, "Do you want to pick door No. 2?" Is it to your advantage to switch your
choice?


Let's try to figure it out. At this point there are two doors, behind one door is a goat 
and behind the other one is a car, thus the probability of choosing the right one is 50\%,
changing the decision doesn't make any difference in the chances of winning. It's pretty
much the same scenario of tossing a coin.

Even though the ideas exposed in the previous paragraph may seem correct, actually 
switching the choice increases the probability of winning. To understand why let's 
first check the facts:

The Monty Hall problem is an interesting puzzle in mathematics inspired by the TV 
game show "Let's make a deal". It's a famous problem that is really easy to understand.

The problem became famous after its appearance in the column "Ask Marilyn" in 1990 and 
it's described below:

Suppose you're on a game show, and you're given the choice of three doors: Behind one 
door is a car; behind the others, goats. You pick a door, say No. 1, and the host, who 
knows what's behind the doors, opens another door, say No. 3, which has a goat. He then 
says to you, "Do you want to pick door No. 2?" Is it to your advantage to switch your
choice?


Let's try to figure it out. At this point there are two doors, behind one door is a goat 
and behind the other one is a car, thus the probability of choosing the right one is 50\%,
changing the decision doesn't make any difference in the chances of winning. It's pretty
much the same scenario of tossing a coin.

Even though the ideas exposed in the previous paragraph may seem correct, actually 
switching the choice increases the probability of winning. To understand why let's 
first check the facts:

\blindtext

\end{document}